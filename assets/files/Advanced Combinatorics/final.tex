\subsection*{Problem 1}
Prove that there is no linear space $(V, E)$ such that $|V | = 50$ and $|e| \in {3, 6, 7}$ for every $e \in E$.

\begin{proof}
\end{proof}

\subsection*{Problem 2}
$H = (V, E)$ is a Steiner triple system on $13$ vertices, $e \in E$. How many edges of $H$ have empty intersection with $e$?

\begin{proof}
\end{proof}

\subsection*{Problem 3}
Assume that the edge set of $K_{2025}$ is colored with $m\ge2$ colors so that the edges in every color form a complete subgraph. Prove that $m\ge2025$.

\begin{proof}
\end{proof}

\subsection*{Problem 4}
A hypergraph $H$ has no three pairwise vertex-disjoint edges. Prove that the greedy algorithm properly colors $H$ with at most 5 colors in any ordering of the vertices.

\begin{proof}
\end{proof}

\subsection*{Problem 5}
Prove that in any 2-coloring of the edges of the complete bipartite graph $K_{41,5}$ there is a monochromatic $K_{3,3}$.

\begin{proof}
\end{proof}

\subsection*{Problem 6}
The edges of a tournament $T_{14}$ are colored with red and blue. Prove that there is a monochromatic transitive triangle.

\begin{proof}
\end{proof}

\subsection*{Problem 7}
A hypergraph with edges $e_{1},e_{2},...,e_{2m}$ is called almost intersecting if \[ e_{1}\cap e_{2}=e_{3}\cap e_{4}=...=e_{2m-1}\cap e_{2m}=\emptyset \] but all other pairs of edges have nonempty intersection. Prove that any almost intersecting $t$-uniform hypergraph satisfies $2m\le\binom{2t}{t}$.

\begin{proof}
\end{proof}

\subsection*{Problem 8}
Show that 10-regular 10-uniform hypergraphs are 2-chromatic.

\begin{proof}
\end{proof}

\subsection*{Problem 9}
A tournament with vertex set $V$ is called $k$-universal if for each $S\subseteq V$ such that $|S|=k$ and for each partition $A\cup B=S$, there exists $v\notin S$ which dominates all vertices of $A$ but no vertices of $B$. Prove that for any fixed $k$ there exist $k$-universal tournaments.

\begin{proof}
\end{proof}

\subsection*{Problem 10}
Assume that two hypergraphs are given on the same $n$-element vertex set, one with edges $e_{i}$, the other is with edges $f_{i}$, $i=1,2,...,m$ and the following holds: $|e_{i}\cap f_{i}|$ is odd for each $i$; $|e_{i}\cap f_{j}|$ is even for all $i\ne j$. Prove that $m\le n$.

\begin{proof}
\end{proof}


