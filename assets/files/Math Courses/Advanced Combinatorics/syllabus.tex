\textbf{Instructor:} András Gyárfás\footnote{Alfréd Rényi Institute of Mathematics, Hungarian Academy of Sciences, Budapest, Hungary}

The course provides a comprehensive introduction to the theory of finite set systems (hypergraphs), exploring both classical foundations and modern developments. Emphasis is placed on fundamental proof techniques, particularly \textit{linear algebraic} and \textit{probabilistic} methods, that have shaped contemporary combinatorics.

\subsection*{Topics}

\begin{enumerate}[1.]
    
    \item \textbf{Basic Concepts:}  
    Incidence matrix, duality, intersection graphs, examples of hypergraphs (paths, cycles, linear spaces, Steiner systems, planar and intersecting linear spaces, finite projective planes).

    \item \textbf{Chromatic Number and Girth:}  
    Proper coloring and the greedy algorithm, extremal examples, classical constructions (Zykov, Mycielski, Tutte, Shift, Kneser graphs), hypergraph constructions via gluing, and the Neszetril-Rödl hypergraph.

    \item \textbf{Ramsey Theory:}  
    Ramsey numbers and their bounds (recursive and non-recursive), exact values, pigeonhole arguments, multicolor and non-diagonal variants, convex $n$-gons in planar point sets; Van der Waerden numbers; Tic-Tac-Toe and the Hales-Jewett theorem (including Shelah's proof); infinite combinatorics and applications.

    \item \textbf{Counting \& Probabilistic Methods:}  
    Proofs by counting (antichains, intersecting hypergraphs, 3-chromatic examples); Erdős's lower bounds for $R(n)$ and $W(k)$; probabilistic constructions, tournaments and transitive subtournaments, paradoxical tournaments (Klein-Szekeres bound); expectations and the probabilistic method in existence proofs; Local Lemma and its applications (Erdős-Lovász theorem, even cycles in regular digraphs); curiosities such as Spencer's injections and “triangle is 2-chromatic.”

    \item \textbf{Linear Algebra Methods:}  
    Dimension bounds (including Oddtown theorem, Fisher inequality), cubic lower bounds for $R(n)$, two-distance sets, cross-intersecting families; homogeneous linear equations in combinatorics (bipartite partitions of $K_n$, discrepancy of hypergraphs); eigenvalue techniques and extremal configurations (Hoffman-Singleton theorem, cages of girth five).

    \item \textbf{Special Topics:}  
    Systems of distinct representatives, chain decompositions, hypergraphs with $n$, $n+1$, and $n+2$ edges; critical 3-colorable hypergraphs; sunflower theorem and sum-free sets; factorizations of $K_n$ and Steiner triple systems.

    \item \textbf{Advanced Topics:}  
    Cyclic generators in Desarguesian planes, hypergraph factorization, two proofs of the Perfect Graph Theorem, constructive super-polynomial lower bounds for $R(n)$, geometric hypergraphs and Borsuk's conjecture, graphs with large chromatic number and girth, Paley graphs and Paley tournaments.

\end{enumerate}
